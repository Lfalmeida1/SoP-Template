\documentclass{article}
\usepackage[T1]{fontenc}
\usepackage[utf8]{inputenc}

% Linespread command allows you to change line spacing for the entire document
\linespread{1.18}

% Tweak page margins
\addtolength{\oddsidemargin}{-.875in}
\addtolength{\evensidemargin}{-.875in}
\addtolength{\textwidth}{1.75in}

\addtolength{\topmargin}{-.875in}
\addtolength{\textheight}{1.75in}

\usepackage{natbib}
\usepackage{hyperref}
\usepackage{xcolor}
\usepackage{xspace}
\usepackage{fancyhdr}
\hypersetup{
    colorlinks,
    linkcolor={red!50!black},
    citecolor={blue!50!black},
    urlcolor={blue!80!black}
}

\newcommand{\HRule}{\rule{\linewidth}{0.5mm}}
\newcommand{\Hrule}{\rule{\linewidth}{0.3mm}}

% Project specific macros
\newcommand{\graphite}{GRAPHITE\xspace}
\newcommand{\wave}{WAVE\xspace}

% School specific macros
\newcommand{\schoolShort}{Stanford\xspace}
\newcommand{\school}{Stanford\xspace}
\newcommand{\schoolLong}{Stanford University\xspace}

\newcommand{\profOne}{Prof. Matei Zaharia\xspace}
\newcommand{\profTwo}{Prof. Christos Kozyrakis\xspace}
\newcommand{\profThree}{Prof. Jure Leskovec\xspace}

% Creates header for each page
\usepackage{fancyhdr}
\pagestyle{fancy}
\fancyhf{}
\fancyhead[LE,RO]{\header\hskip\linepagesep\vfootline\thepage}
\newskip\linepagesep \linepagesep 5pt\relax
\def\vfootline{%
    \begingroup
    	\rule[-10pt]{0.75pt}{25pt}
    \endgroup
}
\def\header{%
	\begin{minipage}[]{120pt}
		\hfill Swapnil Gandhi 		% Applicant Name
    	\par \hfill 				% Formatting boilerplate
    	CS, PhD, Fall 2022 			% Area, Program, Cycle, Year
    \end{minipage}
}
\fancyhead[RE,LO]{Statement of Purpose | \schoolLong}
\renewcommand\headrulewidth{0pt}

\begin{document}

%% Why do you wish to attend graduate school? What would you like to study? Keep it broad, details come-in later

Ever since I decided to pursue a Bachelor's degree in Physics at the age of 17, I have been passionate about research and its potential to solve complex problems. Throughout my undergraduate years, I gravitated toward experimental physics, and by the time I was nearing the end of my degree, I began actively participating in research in the same laboratory where I would later complete my Master's. This early exposure to research not only solidified my interest in the field but also set the stage for my academic and professional journey in materials science.

During my Master's in Materials Science and Engineering, I specialized in the characterization and application of 2D materials, which I find particularly exciting due to their promising potential in various cutting-edge technologies. Specifically, I focused on the development and characterization of nanocomposites, exploring their use in a wide range of devices, including sensors, organic photovoltaics, transistors, and light-emitting diodes (LEDs). One of my major research projects involved studying an aqueous solution nanocomposite made from two commercial materials: the conductive polymer composite PEDOT
and 2D-MoS$_2$. The primary goal of this project was to investigate the potential of this nanocomposite for use in transparent electrodes for organic photovoltaic devices. The combination of these materials offered a unique opportunity to explore the properties of 2D materials in real-world applications.

As a member of a research group focused on nanostructured materials, I was deeply involved in the characterization of the composite. My work involved comprehensive morphological, electrical, and optical characterizations using advanced techniques such as Scanning Electron Microscopy (SEM) and Transmission Electron Microscopy (TEM) for morphological analysis, as well as electrical conductivity measurements and temperature-dependent electrical conductivity to assess the material's electronic properties. For optical characterization, I employed Ultraviolet-Visible (UV-VIS) spectrophotometry and Raman spectroscopy to probe the material's optical properties and confirm its suitability for photovoltaic applications.

The significance of this research was highlighted when I had the opportunity to present my findings at two international conferences—the B-MRS-XXI and B-MRS-XX conferences—as well as at a Brazilian conference focused on carbon nanostructures. This exposure to the global research community not only helped me refine my scientific communication skills but also reinforced my commitment to contributing to advancements in material science. Additionally, my work is currently being developed into a paper that is in the process of submission for publication.

After completing my Master's, I expanded my experience by joining the SENAI Institute for Innovation (ISI) in Brazil, where I worked on the development of lithium-ion batteries. This role allowed me to diversify my research experience, particularly in the field of energy storage. My primary focus was on the synthesis and characterization of a lignin-based anode for use in lithium-ion batteries, a project that required the integration of materials science with sustainable energy solutions. I had the opportunity to further enhance my laboratory skills by learning and applying several new characterization techniques, including Dynamic Light Scattering (DLS), rheology, X-ray diffraction (XRD), and cyclic voltammetry (CV).

At ISI, I also gained invaluable insight into the industrial side of research, which gave me a broader perspective on the challenges and opportunities of scaling up research from the laboratory to production. This experience not only sharpened my practical skills but also deepened my understanding of the real-world applications of advanced materials, and reinforced my commitment to translating fundamental research into tangible innovations.

In addition to my research experience, I have always valued the opportunity to share knowledge and help others grow. During my undergraduate studies, I tutored students in Pre-Calculus for two semesters, which helped me develop strong teaching and communication skills. Later, during my Master's, I had the privilege of working alongside my advisor in an experimental physics class, where I assisted students in conducting experiments and helped explain theoretical concepts. This experience allowed me to deepen my own understanding of the material while honing my ability to explain complex topics in accessible ways.

Looking ahead, I am eager to continue my research on 2D materials and devices, particularly at EPFL, which I believe offers an ideal environment for advanced study in Materials Science and Engineering. The opportunity to work with Professor Andras Kis, whose expertise aligns perfectly with my research interests, would be a tremendous privilege. I have already made contact with him and am excited about the prospect of collaborating with him on cutting-edge projects in nanomaterials and their applications in electronics.

In summary, my academic journey—spanning undergraduate research, a Master’s in Materials Science, and industry experience—has equipped me with a diverse set of skills and a deep understanding of materials science. I am confident that these experiences, combined with my unwavering curiosity and drive to push the boundaries of scientific knowledge, will enable me to succeed in my PhD studies. I am excited about the next chapter in my career and the opportunity to contribute to the growing field of 2D materials at EPFL.
% Add some blank space between text and references
\vspace{0.125in}

% References

% **NOTE**: There are better ways to manage citations in LaTeX, most notably using a bibTeX. I wanted to have greater control on how citations were spaced and formatted and therefore ended up hardcoding them here. Your mileage may wary!

\end{document}

% That's All Folks.

% Best of luck, you got this! :)
