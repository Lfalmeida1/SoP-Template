\documentclass{article}
\usepackage[T1]{fontenc}
\usepackage[utf8]{inputenc}

% Linespread command allows you to change line spacing for the entire document
\linespread{1.18}

% Tweak page margins
\addtolength{\oddsidemargin}{-.875in}
\addtolength{\evensidemargin}{-.875in}
\addtolength{\textwidth}{1.75in}

\addtolength{\topmargin}{-.875in}
\addtolength{\textheight}{1.75in}

\usepackage{natbib}
\usepackage{hyperref}
\usepackage{xcolor}
\usepackage{xspace}
\usepackage{fancyhdr}
\hypersetup{
    colorlinks,
    linkcolor={red!50!black},
    citecolor={blue!50!black},
    urlcolor={blue!80!black}
}

\newcommand{\HRule}{\rule{\linewidth}{0.5mm}}
\newcommand{\Hrule}{\rule{\linewidth}{0.3mm}}

% Project specific macros
\newcommand{\graphite}{GRAPHITE\xspace}
\newcommand{\wave}{WAVE\xspace}

% School specific macros
\newcommand{\schoolShort}{Stanford\xspace}
\newcommand{\school}{Stanford\xspace}
\newcommand{\schoolLong}{Stanford University\xspace}

\newcommand{\profOne}{Prof. Matei Zaharia\xspace}
\newcommand{\profTwo}{Prof. Christos Kozyrakis\xspace}
\newcommand{\profThree}{Prof. Jure Leskovec\xspace}

% Creates header for each page
\usepackage{fancyhdr}
\pagestyle{fancy}
\fancyhf{}
\fancyhead[LE,RO]{\header\hskip\linepagesep\vfootline\thepage}
\newskip\linepagesep \linepagesep 5pt\relax
\def\vfootline{%
    \begingroup
    	\rule[-10pt]{0.75pt}{25pt}
    \endgroup
}
\def\header{%
	\begin{minipage}[]{120pt}
		\hfill Luis Felipe Almeida		% Applicant Name
    	\par \hfill 				% Formatting boilerplate
    	EDMX, PhD, 2025 			% Area, Program, Cycle, Year
    \end{minipage}
}
\fancyhead[RE,LO]{Statement of Objectives | EPFL}
\renewcommand\headrulewidth{0pt}

\begin{document}

%% Why do you wish to attend graduate school? What would you like to study? Keep it broad, details come-in later

Since I was 17 and decided to pursue a Bachelor's degree in Physics at the Federal University of Paraná, I have been passionate about research and its potential to solve complex problems. Coming from a small town and moving to a big city like Curitiba was a significant step. Being far from family was difficult, especially as a Latin American. Still, my motivation to pursue a quality education and to work with research helped me through the challenging times of being away from my loved ones.

In my undergraduate years, experimental physics was of much fascination to me. Towards the end of my degree, I started actively participating in research in the same laboratory where, later on, I did my Master's. This early exposure to research not only set in stone my interest in the field but also set the stage for my future academic and professional journey into material science.

In my Master's in Materials Science and Engineering, I specialized in characterizing and applying 2D materials. It is more thrilling for me because of their promising potential in different state-of-the-art technologies. More precisely, I worked on developing and characterizing nanocomposites, considering applications for devices ranging from sensors, organic photovoltaics, transistors, and light-emitting diodes. I pursued a significant research project on an aqueous solution nanocomposite resulting from two commercial materials: the conductive polymer composite PEDOT:PSS and 2D-MoS$_2$. In this work, the primary focus was to study the suitability of this nanocomposite for use as transparent electrodes in organic photovoltaic devices. Integration with such materials has provided an excellent avenue for exploring the properties of 2D materials toward realistic applications.

Having been part of a research group dealing with nanostructured materials, I actively participated in the characterization of the composite. The characterizations of my work included thorough morphological, electrical, and optical characterization. The advanced techniques employed in this work include SEM and TEM for morphological analysis, electrical conductivity measurement, and temperature-dependent electrical conductivity to test the electric properties of the material. Ultraviolet-Visible spectrophotometry (UV-vis) and Raman spectroscopy were employed for optical characterization. The meaning behind these specific types of characterizations was to apply this material in organic photovoltaics.

The relevance of the present work was pointed out when I had the opportunity to present my findings at two international conferences, B-MRS-XXI and B-MRS-XX, and at one Brazilian conference on carbon nanostructures. The exposure to the global research community helped me hone my scientific communication and further strengthened my commitment to contributing to advances in material science. This work is currently under development into a paper in the submission process phase for publication.

Post-master's, I extended my experience by joining SENAI Institute for Innovation-ISI, Brazil, to develop lithium-ion batteries. This provided an avenue for diversification into energy storage research. My core role in the project was to synthesize and characterize a lignin-based anode for lithium-ion batteries. The project had, in the general framework, the aim of connecting materials science with sustainable energy solutions. Further laboratory experience involved learning and applying several new characterization techniques: DLS, rheology, X-ray diffraction, and cyclic voltammetry.

At ISI, I also had the chance to receive valuable insight into industry related to research, giving a broader perspective on challenges and opportunities regarding scaling up research from the laboratory to production. The experience sharpened not only my practical skills but also furthered my comprehension of advanced materials regarding real-world application and deepened my commitment to translating fundamental research into tangible innovation.

Apart from my research experience, the opportunity to share knowledge and contribute to the growth of others has always meant much to me. During my undergraduate study, I tutored two semesters of students in Pre-Calculus, a process that helped me develop my teaching and communication skills. Later, when taking my Master's, I had a chance to share the same experimental physics class with my academic adviser. In the lab, I could assist students in conducting experiments and explaining theoretical material. This provided an excellent opportunity to reinforce my knowledge while honing my skills in explaining what might seem complicated in more accessible ways.

In the long run, I look forward to continuing my research in 2D materials and devices at EPFL, which offers the best environment for advanced studies in Materials Science and Engineering. I contacted a former EPFL student and learned about their experience, recognition, and opportunities they gained after completing a PhD at EPFL. They shared insights on how the academic community valued their research and how their degree opened doors to prestigious positions and collaborations. 

It would be a great privilege to work with Professor Andras Kis, whose expertise greatly aligns with my research interests. I have already approached him, and I am excited about the opportunity to work with him on state-of-the-art projects in nanomaterials and their applications in electronic devices.

Pursuing a PhD would be a crucial step in achieving my long-term goals, which include contributing to groundbreaking research and advancing humanity's knowledge. I aspire to become a university professor or an active academic within the industry, as both would be an incredible honor and a fulfillment of my professional ambitions.

In short, my academic background is comprehensive, including undergraduate research, a Master's in Materials Science, and industrial experience. It thus prepared me with diverse skills and a deep understanding of materials science. These experiences, combined with my curiosity and drive to advance scientific knowledge, make me confident I will succeed in my PhD studies.

I look forward to this new chapter in my career, which I hope to establish in the fast-growing area of 2D material research at EPFL.
% Add some blank space between text and references
\vspace{0.125in}

% References

% **NOTE**: There are better ways to manage citations in LaTeX, most notably using a bibTeX. I wanted to have greater control on how citations were spaced and formatted and therefore ended up hardcoding them here. Your mileage may wary!

\end{document}

% That's All Folks.

% Best of luck, you got this! :)
