\documentclass{article}
\usepackage[T1]{fontenc}
\usepackage[utf8]{inputenc}

% Linespread command allows you to change line spacing for the entire document
\linespread{1.18}

% Tweak page margins
\addtolength{\oddsidemargin}{-.875in}
\addtolength{\evensidemargin}{-.875in}
\addtolength{\textwidth}{1.75in}

\addtolength{\topmargin}{-.875in}
\addtolength{\textheight}{1.75in}

\usepackage{natbib}
\usepackage{hyperref}
\usepackage{xcolor}
\usepackage{xspace}
\usepackage{fancyhdr}
\hypersetup{
    colorlinks,
    linkcolor={red!50!black},
    citecolor={blue!50!black},
    urlcolor={blue!80!black}
}

\newcommand{\HRule}{\rule{\linewidth}{0.5mm}}
\newcommand{\Hrule}{\rule{\linewidth}{0.3mm}}

% Project specific macros
\newcommand{\graphite}{GRAPHITE\xspace}
\newcommand{\wave}{WAVE\xspace}

% School specific macros
\newcommand{\schoolShort}{Stanford\xspace}
\newcommand{\school}{Stanford\xspace}
\newcommand{\schoolLong}{Stanford University\xspace}

\newcommand{\profOne}{Prof. Matei Zaharia\xspace}
\newcommand{\profTwo}{Prof. Christos Kozyrakis\xspace}
\newcommand{\profThree}{Prof. Jure Leskovec\xspace}

% Creates header for each page
\usepackage{fancyhdr}
\pagestyle{fancy}
\fancyhf{}
\fancyhead[LE,RO]{\header\hskip\linepagesep\vfootline\thepage}
\newskip\linepagesep \linepagesep 5pt\relax
\def\vfootline{%
    \begingroup
    	\rule[-10pt]{0.75pt}{25pt}
    \endgroup
}
\def\header{%
	\begin{minipage}[]{120pt}
		\hfill Swapnil Gandhi 		% Applicant Name
    	\par \hfill 				% Formatting boilerplate
    	CS, PhD, Fall 2022 			% Area, Program, Cycle, Year
    \end{minipage}
}
\fancyhead[RE,LO]{Statement of Purpose | \schoolLong}
\renewcommand\headrulewidth{0pt}

\begin{document}

%% Why do you wish to attend graduate school? What would you like to study? Keep it broad, details come-in later

Ever since I chose to pursue a bachelor's degree in physics at 17, I knew I wanted to work in research. Towards the end of my undergraduate studies, I began conducting research in the same laboratory where I later completed my master's degree. I am interested in the characterization of 2D materials and their application in various devices, including sensors, solar cells, transistors, and LEDs. During my master's in Materials Science and Engineering, I presented my work on the characterization of a nanocomposite at two international conferences: B-MRS-XXI and B-MRS-XX, as well as at a Brazilian conference on carbon nanostructures.

%% Describe 2-3 past projects that might be relevant to your research interests. (10-12 lines per project)

% PROJECT 1: P3 - Distributed Graph Neural Network Training at Scale
During my Master's, I worked with an aqueous solution nanocomposite made from two commercial materials: the polymer composite PEDOT:PSS  and 2D-MoS$_2$. The initial purpose of studying this nanocomposite was to explore its application in the transparent electrode of an organic photovoltaic device. The research group I was part of focused on the application of nanostructured materials in organic photovoltaics and sensors. With this in mind, we conducted morphological, electrical, and optical characterizations.

The techniques used were Scanning Electron Microscopy (SEM) and Transmission Electron Microscopy (TEM) for morphological characterization; electrical conductivity and temperature-dependent electrical conductivity measurements for the electrical characterization; and Ultraviolet-Visible (UV-VIS) spectrophotometry and Raman spectroscopy for the optical characterization.

This work was presented at two international conferences, and in addition, a paper is currently in the process of submission.

% PROJECT 2: SURGEON - Early-Exit Inference

After my Master's, I started working at the SENAI Institute for Innovation (ISI) in Brazil, where I focused on lithium-ion batteries. My work primarily involved the synthesis and characterization of a lignin-based anode. During my time at ISI, I gained experience in additional characterization techniques, such as Dynamic Light Scattering (DLS), rheology, X-ray diffraction (XRD), and cyclic voltammetry (CV).

Working at ISI also allowed me to see the "other side" of research—the industrial side. It was a valuable opportunity to experience firsthand what it’s like to work with research at a larger scale.
% PROJECT 3: Graphite - Distributed Temporal Graph Processing



%% Non-research accomplishments (e.g. Grades, Academic Service, Work experience) (10-12 lines)

% Grades
During my time as a physics student, I tutored for two semesters in the course Pre-Calculus. This experience allowed me to develop my teaching skills by helping students with exercises and assisting them in understanding the subject matter. Additionally, during my Master's, I worked alongside my advisor in one of their classes. It was an experimental physics course, where I assisted students in conducting experiments and also provided theoretical explanations when possible.


%% Why this school? List professors you would like to work with and why? (10-12 Lines)

I believe that my experience with research during my Master's and at the SENAI Institute for Innovation has provided me with a unique perspective on solving problems related to materials research. Additionally, my hands-on experience in the laboratory will be valuable as I pursue my PhD. At EPFL, I would like to continue working with 2D materials and devices. I believe that Professor Andras Kis would be the ideal mentor to help me continue on this path. I have already made contact with him and would be eager to work with him.

%% Summary (3-4 Lines)

In summary, I believe I bring with me research experience, industry-sharpened laboratory skills, and above all, an insatiable desire to learn and excel. I look forward to the next milestone in my life -- a PhD in Materials Science and Engineering from EPFL 

% Add some blank space between text and references
\vspace{0.125in}

% References

% **NOTE**: There are better ways to manage citations in LaTeX, most notably using a bibTeX. I wanted to have greater control on how citations were spaced and formatted and therefore ended up hardcoding them here. Your mileage may wary!

\end{document}

% That's All Folks.

% Best of luck, you got this! :)
